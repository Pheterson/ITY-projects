\documentclass[11pt,a4paper,twocolumn]{article}
\usepackage[left=1.5cm,top=2.5cm,text={18cm,25cm}]{geometry}
\usepackage[IL2]{fontenc}
\usepackage[utf8]{inputenc}
\usepackage[czech]{babel}
\usepackage{blindtext}
\usepackage{amsmath}
\usepackage{amsthm}
\usepackage{amsfonts}
\usepackage{hyperref}
\usepackage{nameref}
\usepackage{setspace}
\usepackage{wasysym}

\theoremstyle{definition}
\newtheorem{definition}{Definice}[]
\newtheorem{sentence}{Věta}[]

\begin{document}

\begin{titlepage}
    \begin{center}
    \huge\textsc{Fakulta informačních technologií\\Vysoké učení technické v Brně}\\
    \renewcommand{\baselinestretch}{0.4}
    \vspace{\stretch{0.382}}
    \Large{Typografie a publikování \--- 2. projekt\\Sazba dokumentů a matematických výrazů}\\
    \vspace{\stretch{0.618}}
    \renewcommand{\baselinestretch}{0.3}
    \end{center}
{\large 2019\hfill Peter Koprda (xkoprd00)}
\end{titlepage}

\newpage

\section*{Úvod}
    \label{sec:Úvod}
    V této úloze si vyzkoušíme sazbu titulní strany, matematických vzorců, prostředí a dalších textových struktur obvyklých pro technicky zaměřené texty (například rovnice~\eqref{eq:1} nebo Definice~\ref{definice1} na straně~\pageref{sec:Úvod}). Pro odkazovaní na vzorce a struktury zásadně používáme příkaz \verb|\label| a \verb|\ref| případně \verb!\pageref! pokud se chceme odkázat na stranu výskytu.\par
    Na titulní straně je využito sázení nadpisu podle optického středu s využitím zlatého středu. Tento postup byl probírán na přednášce. Dále je použito odřádkování se zadanou relativní velikostí 0.4 em a 0.3 em.

\section{Matematický text}
    Nejprve se podíváme na sázení matematických symbolů a výrazů v plynulém textu včetně sazby definic a vět s využitím balíčku \texttt{amsthm}. Rovněž použijeme poznámku pod čarou s použitím příkazu \verb|\footnote|. Někdy je vhodné použít konstrukci \verb|\mbox{}|, která říká, že text nemá být zalomen.
    \begin{definition}
        \label{definice1}
         Zásobníkový automat (ZA) je \emph{definován jako sedmice tvaru} A=(Q,$\scriptstyle\sum$, $\Gamma$, $\delta$, $q_0$, $Z_0$, F), \emph{kde}:
         \begin{itemize}
           \item Q je \emph{konečná množina} vnitřích (řídicích) stavů,
           \item $\sum$ je \emph{konečná} vstupní abeceda,
           \item $\Gamma$ je \emph{konečná} zásobníková abeceda,
           \item $\delta$ je přechodová funkce Q$\times(\scriptstyle\sum\cup\{\epsilon\})\times\Gamma \rightarrow 2^{{Q \times\Gamma}^{\ast}}$,
           \item $q_0$ $\in$ Q je počáteční stav, $Z_0$ $\in$ $\Gamma$ je startovací symbol zásobníku a F $\subseteq$ je \emph{množina} koncových stavů.
         \end{itemize}\par
    \end{definition}
    Nechť P=(Q,$\scriptstyle\sum$, $\Gamma$, $\delta$, $q_0$, $Z_0$, F) je zásobníkový automat. \emph{Konfigurací} nazveme trojici (q,$\omega$,$\alpha$) $\in$ Q$\times\scriptstyle\sum^{\ast}\times\Gamma^{\ast}$, kde \emph{q} je aktuální stav vnitřního řízení, $\omega$ je dosud nezpracovaná část vstupního řetězce a $\alpha$ = $Z_{i_1}Z_{i_2}$\dots$Z_{i_k}$ je obsah zásobníku\footnotemark.
    \footnotetext{$Z_{i_1}$ je vrchol zásobníku}

    \subsection{Podsekce obsahující větu a odkaz}
        \begin{definition}
            \label{definition2}
            Řetězec $\omega$ nad abecedou $\scriptstyle\sum$ je přijat ZA A \emph{jestliže} ($q_0$,$\omega$,$Z_0$) $\overset{\ast}{\underset{A}{\vdash}}$ ($q_F$,$\epsilon$,$\gamma$) \emph{pro nějaké $\gamma$ $\in$ $\Gamma^\ast$ a $q_F$ $\in$ F. Množinu L(A)=$\{\omega$ $|$ $\omega$ je přijat ZA A$\}$} $\subseteq$ $\scriptstyle\sum^\ast$ \emph{nazýváme} jazyk přijímaný TS \emph{M}.
        \end{definition}
        Nyní si vyzkoušíme sazbu vět a důkazů opět s použitím balíku \texttt{amsmath}.
        \begin{sentence}
            \label{sentence}
            \emph{Třída jazyků, které jsou přijímány ZA, odpovídá} bezkontextovým jazykům.
        \end{sentence}

        \vspace{-1.0em}
        \begin{proof}
            V důkaze vyjdeme z Definice~\ref{definice1} a~\ref{definition2}.
        \end{proof}


\section{Rovnice a odkazy}
    Složitejší matematické formulace sázíme mimo plynulý text. Lze umístit několik výrazů na jeden řádek, ale pak je třeba tyto vhodné oddělit, například příkazem \verb!\quad!.\\[1.0em]
    $\sqrt[i]{x_i^3}$ kde $x_i$ je i-té sudé číslo splňujúci $x_i^{2-x_i^{i^2}}$ $\leq$ $x_i^{y_i^3}$\par
    \vspace{0.3em}
    V rovnici (1) jsou využity tři typy závorek s různou explicitně definovanou velikostí.
    \begin{eqnarray}
    \label{eq:1}
        x & = & \Bigg[\bigg\{\Big[a+b\Big]\ast c\bigg\}^d \ominus 1\Bigg]^{1/_2}\\
        y & = & \lim\limits_{x \to \infty} \frac{\frac{1}{\log_{10} x}}{sin^2x+cos^2x} \nonumber
    \end{eqnarray}

    V této větě vidíme, jak vypadá implicitní vysázení limity $\lim_{n \to \infty} f{(\emph{n})}$ v normálním odstavci textu. Podobně je to i s dalšími symboly jako $\prod_{i=1}^{n} 2^i$ či $\bigcap_{A\in B}A$. V případě vzorců $\lim\limits_{n \to \infty} f{(\emph{n})}$ a $\displaystyle\prod_{i=1}^{n} 2^i$ jsme si vynutili méně úspornou sazbu příkazem \verb!\limits!.

    \begin{equation}
        \int_{b}^{a} g(x)dx\ =\ - \int\limits_{a}^{b} f(x) dx
    \end{equation}

    \begin{equation}
        \overline{\overline{A\wedge B}} \ \Leftrightarrow \ \overline{\overline{A}\vee \overline{B}}
    \end{equation}
    
\section{Matice}
    Pro sázení matic se velmi často používá prostředí \texttt{array} a závorky (\verb!\left!,\verb!\right!).

    \[
    \begin{bmatrix}
      \  & \widehat{\beta + \gamma} & \hat{\pi}\ \\[0.3em]
      \ \vec{a} & \overleftrightarrow{AC} &\
    \end{bmatrix}=1 \Longleftrightarrow \mathbb{Q} =\textbf{R}
    \]


    \textbf{A}\ =
    $\begin{vmatrix}
      \ a_{11} & a_{12} & \cdots & a_{1n}\ \\
      \ a_{21} & a_{22} & \cdots & a_{2n}\ \\
      \ \vdots & \vdots & \ddots & \vdots\ \\
      \ a_{m1} & a_{m2} & \cdots & a_{mn}\
    \end{vmatrix}=\
    \begin{matrix}
      t & u \\
      v & w
    \end{matrix}\ =tw-uv$\\

    Prostředí \texttt{array} lze úspešně využít i jinde.

    \begin{equation*}
      \left(\!
        \begin{array}{c}
          n\\
          k
        \end{array}
    \!\right)=
    \begin{cases}
      \ 0 & \quad\text{pro } k < 0 \text{ nebo } k > n \\
      \ \frac{n!}{k!(n-k)!} & \quad\text{pro } 0\leq k\leq n
    \end{cases}
    \end{equation*}

\end{document} 